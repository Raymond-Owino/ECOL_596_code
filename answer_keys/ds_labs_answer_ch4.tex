% Options for packages loaded elsewhere
\PassOptionsToPackage{unicode}{hyperref}
\PassOptionsToPackage{hyphens}{url}
%
\documentclass[
]{article}
\usepackage{amsmath,amssymb}
\usepackage{iftex}
\ifPDFTeX
  \usepackage[T1]{fontenc}
  \usepackage[utf8]{inputenc}
  \usepackage{textcomp} % provide euro and other symbols
\else % if luatex or xetex
  \usepackage{unicode-math} % this also loads fontspec
  \defaultfontfeatures{Scale=MatchLowercase}
  \defaultfontfeatures[\rmfamily]{Ligatures=TeX,Scale=1}
\fi
\usepackage{lmodern}
\ifPDFTeX\else
  % xetex/luatex font selection
\fi
% Use upquote if available, for straight quotes in verbatim environments
\IfFileExists{upquote.sty}{\usepackage{upquote}}{}
\IfFileExists{microtype.sty}{% use microtype if available
  \usepackage[]{microtype}
  \UseMicrotypeSet[protrusion]{basicmath} % disable protrusion for tt fonts
}{}
\makeatletter
\@ifundefined{KOMAClassName}{% if non-KOMA class
  \IfFileExists{parskip.sty}{%
    \usepackage{parskip}
  }{% else
    \setlength{\parindent}{0pt}
    \setlength{\parskip}{6pt plus 2pt minus 1pt}}
}{% if KOMA class
  \KOMAoptions{parskip=half}}
\makeatother
\usepackage{xcolor}
\usepackage[margin=1in]{geometry}
\usepackage{color}
\usepackage{fancyvrb}
\newcommand{\VerbBar}{|}
\newcommand{\VERB}{\Verb[commandchars=\\\{\}]}
\DefineVerbatimEnvironment{Highlighting}{Verbatim}{commandchars=\\\{\}}
% Add ',fontsize=\small' for more characters per line
\usepackage{framed}
\definecolor{shadecolor}{RGB}{248,248,248}
\newenvironment{Shaded}{\begin{snugshade}}{\end{snugshade}}
\newcommand{\AlertTok}[1]{\textcolor[rgb]{0.94,0.16,0.16}{#1}}
\newcommand{\AnnotationTok}[1]{\textcolor[rgb]{0.56,0.35,0.01}{\textbf{\textit{#1}}}}
\newcommand{\AttributeTok}[1]{\textcolor[rgb]{0.13,0.29,0.53}{#1}}
\newcommand{\BaseNTok}[1]{\textcolor[rgb]{0.00,0.00,0.81}{#1}}
\newcommand{\BuiltInTok}[1]{#1}
\newcommand{\CharTok}[1]{\textcolor[rgb]{0.31,0.60,0.02}{#1}}
\newcommand{\CommentTok}[1]{\textcolor[rgb]{0.56,0.35,0.01}{\textit{#1}}}
\newcommand{\CommentVarTok}[1]{\textcolor[rgb]{0.56,0.35,0.01}{\textbf{\textit{#1}}}}
\newcommand{\ConstantTok}[1]{\textcolor[rgb]{0.56,0.35,0.01}{#1}}
\newcommand{\ControlFlowTok}[1]{\textcolor[rgb]{0.13,0.29,0.53}{\textbf{#1}}}
\newcommand{\DataTypeTok}[1]{\textcolor[rgb]{0.13,0.29,0.53}{#1}}
\newcommand{\DecValTok}[1]{\textcolor[rgb]{0.00,0.00,0.81}{#1}}
\newcommand{\DocumentationTok}[1]{\textcolor[rgb]{0.56,0.35,0.01}{\textbf{\textit{#1}}}}
\newcommand{\ErrorTok}[1]{\textcolor[rgb]{0.64,0.00,0.00}{\textbf{#1}}}
\newcommand{\ExtensionTok}[1]{#1}
\newcommand{\FloatTok}[1]{\textcolor[rgb]{0.00,0.00,0.81}{#1}}
\newcommand{\FunctionTok}[1]{\textcolor[rgb]{0.13,0.29,0.53}{\textbf{#1}}}
\newcommand{\ImportTok}[1]{#1}
\newcommand{\InformationTok}[1]{\textcolor[rgb]{0.56,0.35,0.01}{\textbf{\textit{#1}}}}
\newcommand{\KeywordTok}[1]{\textcolor[rgb]{0.13,0.29,0.53}{\textbf{#1}}}
\newcommand{\NormalTok}[1]{#1}
\newcommand{\OperatorTok}[1]{\textcolor[rgb]{0.81,0.36,0.00}{\textbf{#1}}}
\newcommand{\OtherTok}[1]{\textcolor[rgb]{0.56,0.35,0.01}{#1}}
\newcommand{\PreprocessorTok}[1]{\textcolor[rgb]{0.56,0.35,0.01}{\textit{#1}}}
\newcommand{\RegionMarkerTok}[1]{#1}
\newcommand{\SpecialCharTok}[1]{\textcolor[rgb]{0.81,0.36,0.00}{\textbf{#1}}}
\newcommand{\SpecialStringTok}[1]{\textcolor[rgb]{0.31,0.60,0.02}{#1}}
\newcommand{\StringTok}[1]{\textcolor[rgb]{0.31,0.60,0.02}{#1}}
\newcommand{\VariableTok}[1]{\textcolor[rgb]{0.00,0.00,0.00}{#1}}
\newcommand{\VerbatimStringTok}[1]{\textcolor[rgb]{0.31,0.60,0.02}{#1}}
\newcommand{\WarningTok}[1]{\textcolor[rgb]{0.56,0.35,0.01}{\textbf{\textit{#1}}}}
\usepackage{graphicx}
\makeatletter
\def\maxwidth{\ifdim\Gin@nat@width>\linewidth\linewidth\else\Gin@nat@width\fi}
\def\maxheight{\ifdim\Gin@nat@height>\textheight\textheight\else\Gin@nat@height\fi}
\makeatother
% Scale images if necessary, so that they will not overflow the page
% margins by default, and it is still possible to overwrite the defaults
% using explicit options in \includegraphics[width, height, ...]{}
\setkeys{Gin}{width=\maxwidth,height=\maxheight,keepaspectratio}
% Set default figure placement to htbp
\makeatletter
\def\fps@figure{htbp}
\makeatother
\setlength{\emergencystretch}{3em} % prevent overfull lines
\providecommand{\tightlist}{%
  \setlength{\itemsep}{0pt}\setlength{\parskip}{0pt}}
\setcounter{secnumdepth}{-\maxdimen} % remove section numbering
\ifLuaTeX
  \usepackage{selnolig}  % disable illegal ligatures
\fi
\IfFileExists{bookmark.sty}{\usepackage{bookmark}}{\usepackage{hyperref}}
\IfFileExists{xurl.sty}{\usepackage{xurl}}{} % add URL line breaks if available
\urlstyle{same}
\hypersetup{
  pdftitle={Dslabs Answer Key Ch. 4},
  pdfauthor={Sabrina McNew},
  hidelinks,
  pdfcreator={LaTeX via pandoc}}

\title{Dslabs Answer Key Ch. 4}
\author{Sabrina McNew}
\date{2023-09-07}

\begin{document}
\maketitle

\hypertarget{load-our-libraries-and-set-plot-theme}{%
\subsubsection{Load our libraries and set plot
theme}\label{load-our-libraries-and-set-plot-theme}}

\begin{Shaded}
\begin{Highlighting}[]
\FunctionTok{library}\NormalTok{(dslabs)}
\FunctionTok{library}\NormalTok{(ggthemr)}
\FunctionTok{library}\NormalTok{(tidyr)}
\FunctionTok{library}\NormalTok{(dplyr)}

\FunctionTok{ggthemr}\NormalTok{(}\AttributeTok{palette =} \StringTok{"flat"}\NormalTok{, }\AttributeTok{layout =} \StringTok{"clean"}\NormalTok{, }\AttributeTok{text\_size =} \DecValTok{22}\NormalTok{)}
\end{Highlighting}
\end{Shaded}

\hypertarget{dslabs-introduction-to-data-science-book-first-edition}{%
\subsection{Dslabs: Introduction to Data Science Book First
edition}\label{dslabs-introduction-to-data-science-book-first-edition}}

\hypertarget{rafael-irizarry}{%
\subsubsection{Rafael Irizarry}\label{rafael-irizarry}}

\hypertarget{chapter-4}{%
\subsection{Chapter 4}\label{chapter-4}}

\hypertarget{exercises}{%
\subsubsection{4.2 Exercises}\label{exercises}}

\begin{enumerate}
\def\labelenumi{\arabic{enumi}.}
\tightlist
\item
  Examine the built-in dataset co2. Which of the following is true:
\item
  Examine the built-in dataset ChickWeight. Which of the following is
  true:
\item
  Examine the built-in dataset BOD. Which of the following is true:
\item
  Which of the following built-in datasets is tidy (you can pick more
  than one):
\end{enumerate}

\begin{Shaded}
\begin{Highlighting}[]
\FunctionTok{head}\NormalTok{(co2) }\CommentTok{\# vector of numbers}
\end{Highlighting}
\end{Shaded}

\begin{verbatim}
## [1] 315.42 316.31 316.50 317.56 318.13 318.00
\end{verbatim}

\begin{Shaded}
\begin{Highlighting}[]
\FunctionTok{head}\NormalTok{(ChickWeight)}
\end{Highlighting}
\end{Shaded}

\begin{verbatim}
##   weight Time Chick Diet
## 1     42    0     1    1
## 2     51    2     1    1
## 3     59    4     1    1
## 4     64    6     1    1
## 5     76    8     1    1
## 6     93   10     1    1
\end{verbatim}

\begin{Shaded}
\begin{Highlighting}[]
\FunctionTok{head}\NormalTok{(BOD)}
\end{Highlighting}
\end{Shaded}

\begin{verbatim}
##   Time demand
## 1    1    8.3
## 2    2   10.3
## 3    3   19.0
## 4    4   16.0
## 5    5   15.6
## 6    7   19.8
\end{verbatim}

Sabrina's notes:\\
1. I guess this is \emph{b}; it's just a vector so I wouldn't say it's
``tidy.'' I don't think you really need a character column to be tidy,
but if it were a dataframe it might be tidier. Not sure about this q
tbh.

\begin{enumerate}
\def\labelenumi{\arabic{enumi}.}
\setcounter{enumi}{1}
\item
  \emph{b} This seems pretty tidy. It's ok that individual chicks have
  more than one measurement.
\item
  \emph{c} looks fine
\item
  Ones that are not tidy:\\
  BJSales, is just a vector, not sure what other data relate to it.
  EuStockMarkets: some sort of strange class (``time series'') seems
  more like a matrix. I'm going with untidy, though probably someone
  wanted it in this format UCBAdmissions: terrible format; List of
  tables, not good.
\end{enumerate}

\hypertarget{exercises-1}{%
\subsubsection{4.4 Exercises}\label{exercises-1}}

\begin{enumerate}
\def\labelenumi{\arabic{enumi}.}
\tightlist
\item
  Use the function mutate to add a murders column named rate with the
  per 100,000 murder rate as in the example code above. Make sure you
  redefine murders as done in the example code above ( murders
  \textless- {[}your code{]}) so we can keep using this variable.
\end{enumerate}

\begin{Shaded}
\begin{Highlighting}[]
\NormalTok{murders }\OtherTok{\textless{}{-}} \FunctionTok{mutate}\NormalTok{(murders, }\AttributeTok{rate =}\NormalTok{ total }\SpecialCharTok{/}\NormalTok{ (population}\SpecialCharTok{/}\DecValTok{100000}\NormalTok{))}
\end{Highlighting}
\end{Shaded}

\begin{enumerate}
\def\labelenumi{\arabic{enumi}.}
\setcounter{enumi}{1}
\tightlist
\item
  If rank(x) gives you the ranks of x from lowest to highest, rank(-x)
  gives you the ranks from highest to lowest. Use the function mutate to
  add a column rank containing the rank, from highest to lowest murder
  rate. Make sure you redefine murders so we can keep using this
  variable.
\end{enumerate}

\begin{Shaded}
\begin{Highlighting}[]
\NormalTok{murders }\OtherTok{\textless{}{-}} \FunctionTok{mutate}\NormalTok{(murders, }\AttributeTok{rank =} \FunctionTok{rank}\NormalTok{(}\SpecialCharTok{{-}}\NormalTok{rate))}
\end{Highlighting}
\end{Shaded}

\begin{enumerate}
\def\labelenumi{\arabic{enumi}.}
\setcounter{enumi}{2}
\tightlist
\item
  Use select to show the state names and abbreviations in murders. Do
  not redefine murders, just show the results.
\end{enumerate}

\begin{Shaded}
\begin{Highlighting}[]
\NormalTok{murders }\SpecialCharTok{|\textgreater{}} \FunctionTok{select}\NormalTok{(state, abb) }\SpecialCharTok{\%\textgreater{}\%}\NormalTok{ head}
\end{Highlighting}
\end{Shaded}

\begin{verbatim}
##        state abb
## 1    Alabama  AL
## 2     Alaska  AK
## 3    Arizona  AZ
## 4   Arkansas  AR
## 5 California  CA
## 6   Colorado  CO
\end{verbatim}

\begin{enumerate}
\def\labelenumi{\arabic{enumi}.}
\setcounter{enumi}{3}
\tightlist
\item
  Use filter to show the top 5 states with the highest murder rates.
  After we add murder rate and rank, do not change the murders dataset,
  just show the result. Remember that you can filter based on the rank
  column.
\end{enumerate}

\begin{Shaded}
\begin{Highlighting}[]
\NormalTok{murders }\SpecialCharTok{|\textgreater{}} \FunctionTok{filter}\NormalTok{(rank }\SpecialCharTok{\%in\%} \FunctionTok{c}\NormalTok{(}\DecValTok{1}\SpecialCharTok{:}\DecValTok{5}\NormalTok{))}
\end{Highlighting}
\end{Shaded}

\begin{verbatim}
##                  state abb        region population total      rate rank
## 1 District of Columbia  DC         South     601723    99 16.452753    1
## 2            Louisiana  LA         South    4533372   351  7.742581    2
## 3             Maryland  MD         South    5773552   293  5.074866    4
## 4             Missouri  MO North Central    5988927   321  5.359892    3
## 5       South Carolina  SC         South    4625364   207  4.475323    5
\end{verbatim}

\begin{enumerate}
\def\labelenumi{\arabic{enumi}.}
\setcounter{enumi}{4}
\tightlist
\item
  Create a new data frame called no\_south that removes states from the
  South region. How many states are in this category? You can use the
  function nrow for this. Sabrina Note: to save time/coding lines I'm
  opting just to pipe the result to nrow() to get the number rather than
  creating a new data farme and assigning it to the namespace
\end{enumerate}

\begin{Shaded}
\begin{Highlighting}[]
\NormalTok{murders }\SpecialCharTok{|\textgreater{}} \FunctionTok{filter}\NormalTok{(region }\SpecialCharTok{!=} \StringTok{"South"}\NormalTok{) }\SpecialCharTok{|\textgreater{}} \FunctionTok{nrow}\NormalTok{()}
\end{Highlighting}
\end{Shaded}

\begin{verbatim}
## [1] 34
\end{verbatim}

\begin{enumerate}
\def\labelenumi{\arabic{enumi}.}
\setcounter{enumi}{5}
\tightlist
\item
  Create a new data frame called murders\_nw with only the states from
  the Northeast and the West. How many states are in this category?
\end{enumerate}

\begin{Shaded}
\begin{Highlighting}[]
\NormalTok{murders }\SpecialCharTok{|\textgreater{}} \FunctionTok{filter}\NormalTok{(region }\SpecialCharTok{\%in\%} \FunctionTok{c}\NormalTok{(}\StringTok{"Northwest"}\NormalTok{, }\StringTok{"West"}\NormalTok{)) }\SpecialCharTok{|\textgreater{}} \FunctionTok{nrow}\NormalTok{()}
\end{Highlighting}
\end{Shaded}

\begin{verbatim}
## [1] 13
\end{verbatim}

\begin{enumerate}
\def\labelenumi{\arabic{enumi}.}
\setcounter{enumi}{6}
\tightlist
\item
  Create a table called my\_states that contains rows for states
  satisfying both the conditions: it is in the Northeast or West and the
  murder rate is less than 1. Use select to show only the state name,
  the rate, and the rank.
\end{enumerate}

\begin{Shaded}
\begin{Highlighting}[]
\NormalTok{murders }\SpecialCharTok{|\textgreater{}} \FunctionTok{filter}\NormalTok{(rate }\SpecialCharTok{\textless{}} \DecValTok{1} \SpecialCharTok{\&}\NormalTok{ region }\SpecialCharTok{\%in\%} \FunctionTok{c}\NormalTok{(}\StringTok{"Northwest"}\NormalTok{, }\StringTok{"West"}\NormalTok{)) }\SpecialCharTok{|\textgreater{}}
  \FunctionTok{select}\NormalTok{(state, rate, rank)}
\end{Highlighting}
\end{Shaded}

\begin{verbatim}
##     state      rate rank
## 1  Hawaii 0.5145920   49
## 2   Idaho 0.7655102   46
## 3  Oregon 0.9396843   42
## 4    Utah 0.7959810   45
## 5 Wyoming 0.8871131   43
\end{verbatim}

\hypertarget{exercises-2}{%
\subsubsection{4.6 Exercises}\label{exercises-2}}

1.Repeat the previous exercise, but now instead of creating a new
object, show the result and only include the state, rate, and rank
columns. Use a pipe \textbar\textgreater{} to do this in just one line.
(Answer: see above, already piped for convenience)

\begin{enumerate}
\def\labelenumi{\arabic{enumi}.}
\setcounter{enumi}{1}
\tightlist
\item
  Reset murders to the original table by using data(murders). Use a pipe
  to create a new data frame called my\_states that considers only
  states in the Northeast or West which have a murder rate lower than 1,
  and contains only the state, rate and rank columns. The pipe should
  also have four components separated by three \textbar\textgreater. The
  code should look something like this:
\end{enumerate}

\begin{Shaded}
\begin{Highlighting}[]
\FunctionTok{data}\NormalTok{(murders)}
\NormalTok{my\_states }\OtherTok{\textless{}{-}}\NormalTok{ murders }\SpecialCharTok{|\textgreater{}} 
  \FunctionTok{mutate}\NormalTok{(}\AttributeTok{rate =}\NormalTok{ total }\SpecialCharTok{/}\NormalTok{ (population}\SpecialCharTok{/}\DecValTok{100000}\NormalTok{),}
         \AttributeTok{rank =} \FunctionTok{rank}\NormalTok{(}\SpecialCharTok{{-}}\NormalTok{rate)) }\SpecialCharTok{|\textgreater{}} 
  \FunctionTok{filter}\NormalTok{(region }\SpecialCharTok{\%in\%} \FunctionTok{c}\NormalTok{(}\StringTok{"Northeast"}\NormalTok{, }\StringTok{"West"}\NormalTok{)) }\SpecialCharTok{|\textgreater{}} 
  \FunctionTok{filter}\NormalTok{(rate }\SpecialCharTok{\textless{}} \DecValTok{1}\NormalTok{) }\SpecialCharTok{|\textgreater{}} 
  \FunctionTok{select}\NormalTok{(state, rate, rank)}
\CommentTok{\# Sabrina note, I personally like to separate my filters but YMMV}
\end{Highlighting}
\end{Shaded}

\hypertarget{exercises-3}{%
\subsubsection{4.10 Exercises}\label{exercises-3}}

\begin{Shaded}
\begin{Highlighting}[]
\CommentTok{\#install.packages("NHANES")}
\FunctionTok{library}\NormalTok{(NHANES)}
\FunctionTok{data}\NormalTok{(NHANES)}
\end{Highlighting}
\end{Shaded}

\begin{enumerate}
\def\labelenumi{\arabic{enumi}.}
\tightlist
\item
  We will provide some basic facts about blood pressure. First let's
  select a group to set the standard. We will use 20-to-29-year-old
  females. AgeDecade is a categorical variable with these ages. Note
  that the category is coded like '' 20-29'', with a space in front!
  What is the average and standard deviation of systolic blood pressure
  as saved in the BPSysAve variable? Save it to a variable called ref.
\end{enumerate}

Sabrina's notes: I interpreted ref should be the systolic blood pressure
vector

\begin{Shaded}
\begin{Highlighting}[]
\FunctionTok{head}\NormalTok{(NHANES) }\CommentTok{\#let\textquotesingle{}s have a look }
\end{Highlighting}
\end{Shaded}

\begin{verbatim}
## # A tibble: 6 x 76
##      ID SurveyYr Gender   Age AgeDecade AgeMonths Race1 Race3 Education   
##   <int> <fct>    <fct>  <int> <fct>         <int> <fct> <fct> <fct>       
## 1 51624 2009_10  male      34 " 30-39"        409 White <NA>  High School 
## 2 51624 2009_10  male      34 " 30-39"        409 White <NA>  High School 
## 3 51624 2009_10  male      34 " 30-39"        409 White <NA>  High School 
## 4 51625 2009_10  male       4 " 0-9"           49 Other <NA>  <NA>        
## 5 51630 2009_10  female    49 " 40-49"        596 White <NA>  Some College
## 6 51638 2009_10  male       9 " 0-9"          115 White <NA>  <NA>        
## # i 67 more variables: MaritalStatus <fct>, HHIncome <fct>, HHIncomeMid <int>,
## #   Poverty <dbl>, HomeRooms <int>, HomeOwn <fct>, Work <fct>, Weight <dbl>,
## #   Length <dbl>, HeadCirc <dbl>, Height <dbl>, BMI <dbl>,
## #   BMICatUnder20yrs <fct>, BMI_WHO <fct>, Pulse <int>, BPSysAve <int>,
## #   BPDiaAve <int>, BPSys1 <int>, BPDia1 <int>, BPSys2 <int>, BPDia2 <int>,
## #   BPSys3 <int>, BPDia3 <int>, Testosterone <dbl>, DirectChol <dbl>,
## #   TotChol <dbl>, UrineVol1 <int>, UrineFlow1 <dbl>, UrineVol2 <int>, ...
\end{verbatim}

\begin{Shaded}
\begin{Highlighting}[]
\NormalTok{ref }\OtherTok{\textless{}{-}}\NormalTok{ NHANES }\SpecialCharTok{|\textgreater{}} 
      \FunctionTok{filter}\NormalTok{(AgeDecade }\SpecialCharTok{==} \StringTok{" 20{-}29"} \SpecialCharTok{\&}\NormalTok{ Gender }\SpecialCharTok{==} \StringTok{"female"}\NormalTok{)  }\SpecialCharTok{|\textgreater{}} 
      \FunctionTok{pull}\NormalTok{(BPSysAve)}
\FunctionTok{mean}\NormalTok{(ref, }\AttributeTok{na.rm =}\NormalTok{ T)}
\end{Highlighting}
\end{Shaded}

\begin{verbatim}
## [1] 108.4224
\end{verbatim}

\begin{Shaded}
\begin{Highlighting}[]
\FunctionTok{sd}\NormalTok{(ref, }\AttributeTok{na.rm =}\NormalTok{ T)}
\end{Highlighting}
\end{Shaded}

\begin{verbatim}
## [1] 10.14668
\end{verbatim}

\begin{enumerate}
\def\labelenumi{\arabic{enumi}.}
\setcounter{enumi}{1}
\tightlist
\item
  Using a pipe, assign the average to a numeric variable ref\_avg. Hint:
  Use the code similar to above and then pull.
\end{enumerate}

\begin{Shaded}
\begin{Highlighting}[]
\NormalTok{ref\_avg }\OtherTok{\textless{}{-}}\NormalTok{ NHANES }\SpecialCharTok{|\textgreater{}} 
      \FunctionTok{filter}\NormalTok{(AgeDecade }\SpecialCharTok{==} \StringTok{" 20{-}29"} \SpecialCharTok{\&}\NormalTok{ Gender }\SpecialCharTok{==} \StringTok{"female"}\NormalTok{)  }\SpecialCharTok{|\textgreater{}} 
      \FunctionTok{pull}\NormalTok{(BPSysAve) }\SpecialCharTok{|\textgreater{}} 
      \FunctionTok{mean}\NormalTok{(}\AttributeTok{na.rm =}\NormalTok{ T)}
\NormalTok{ref\_avg}
\end{Highlighting}
\end{Shaded}

\begin{verbatim}
## [1] 108.4224
\end{verbatim}

\begin{enumerate}
\def\labelenumi{\arabic{enumi}.}
\setcounter{enumi}{2}
\tightlist
\item
  Now report the min and max values for the same group.
\end{enumerate}

\begin{Shaded}
\begin{Highlighting}[]
\NormalTok{NHANES }\SpecialCharTok{|\textgreater{}} 
      \FunctionTok{filter}\NormalTok{(AgeDecade }\SpecialCharTok{==} \StringTok{" 20{-}29"} \SpecialCharTok{\&}\NormalTok{ Gender }\SpecialCharTok{==} \StringTok{"female"}\NormalTok{)  }\SpecialCharTok{|\textgreater{}} 
      \FunctionTok{pull}\NormalTok{(BPSysAve) }\SpecialCharTok{|\textgreater{}} \FunctionTok{range}\NormalTok{(}\AttributeTok{na.rm =}\NormalTok{ T)}
\end{Highlighting}
\end{Shaded}

\begin{verbatim}
## [1]  84 179
\end{verbatim}

\begin{enumerate}
\def\labelenumi{\arabic{enumi}.}
\setcounter{enumi}{3}
\tightlist
\item
  Compute the average and standard deviation for females, but for each
  age group separately rather than a selected decade as in question 1.
  Note that the age groups are defined by AgeDecade. Hint: rather than
  filtering by age and gender, filter by Gender and then use group\_by.
\end{enumerate}

\begin{Shaded}
\begin{Highlighting}[]
\NormalTok{NHANES }\SpecialCharTok{|\textgreater{}} 
  \FunctionTok{filter}\NormalTok{(Gender }\SpecialCharTok{==} \StringTok{"female"}\NormalTok{) }\SpecialCharTok{|\textgreater{}} 
  \FunctionTok{group\_by}\NormalTok{(AgeDecade) }\SpecialCharTok{|\textgreater{}} 
  \FunctionTok{summarize}\NormalTok{(}\AttributeTok{mean\_bps =} \FunctionTok{mean}\NormalTok{(BPSysAve, }\AttributeTok{na.rm =}\NormalTok{ T),}
            \AttributeTok{sd\_bps =} \FunctionTok{sd}\NormalTok{(BPSysAve, }\AttributeTok{na.rm =}\NormalTok{ T))}
\end{Highlighting}
\end{Shaded}

\begin{verbatim}
## # A tibble: 9 x 3
##   AgeDecade mean_bps sd_bps
##   <fct>        <dbl>  <dbl>
## 1 " 0-9"        100.   9.07
## 2 " 10-19"      104.   9.46
## 3 " 20-29"      108.  10.1 
## 4 " 30-39"      111.  12.3 
## 5 " 40-49"      115.  14.5 
## 6 " 50-59"      122.  16.2 
## 7 " 60-69"      127.  17.1 
## 8 " 70+"        134.  19.8 
## 9  <NA>         142.  22.9
\end{verbatim}

\begin{enumerate}
\def\labelenumi{\arabic{enumi}.}
\setcounter{enumi}{4}
\tightlist
\item
  Repeat exercise 4 for males.
\end{enumerate}

\begin{Shaded}
\begin{Highlighting}[]
\NormalTok{NHANES }\SpecialCharTok{|\textgreater{}} 
  \FunctionTok{filter}\NormalTok{(Gender }\SpecialCharTok{==} \StringTok{"male"}\NormalTok{) }\SpecialCharTok{|\textgreater{}} 
  \FunctionTok{group\_by}\NormalTok{(AgeDecade) }\SpecialCharTok{|\textgreater{}} 
  \FunctionTok{summarize}\NormalTok{(}\AttributeTok{mean\_bps =} \FunctionTok{mean}\NormalTok{(BPSysAve, }\AttributeTok{na.rm =}\NormalTok{ T),}
            \AttributeTok{sd\_bps =} \FunctionTok{sd}\NormalTok{(BPSysAve, }\AttributeTok{na.rm =}\NormalTok{ T))}
\end{Highlighting}
\end{Shaded}

\begin{verbatim}
## # A tibble: 9 x 3
##   AgeDecade mean_bps sd_bps
##   <fct>        <dbl>  <dbl>
## 1 " 0-9"        97.4   8.32
## 2 " 10-19"     110.   11.2 
## 3 " 20-29"     118.   11.3 
## 4 " 30-39"     119.   12.3 
## 5 " 40-49"     121.   14.0 
## 6 " 50-59"     126.   17.8 
## 7 " 60-69"     127.   17.5 
## 8 " 70+"       130.   18.7 
## 9  <NA>        136.   23.5
\end{verbatim}

\begin{enumerate}
\def\labelenumi{\arabic{enumi}.}
\setcounter{enumi}{5}
\tightlist
\item
  We can actually combine both summaries for exercises 4 and 5 into one
  line of code. This is because group\_by permits us to group by more
  than one variable. Obtain one big summary table using
  group\_by(AgeDecade, Gender).
\end{enumerate}

\begin{Shaded}
\begin{Highlighting}[]
\NormalTok{NHANES }\SpecialCharTok{|\textgreater{}} 
  \FunctionTok{group\_by}\NormalTok{(AgeDecade, Gender) }\SpecialCharTok{|\textgreater{}} 
  \FunctionTok{summarize}\NormalTok{(}\AttributeTok{mean\_bps =} \FunctionTok{mean}\NormalTok{(BPSysAve, }\AttributeTok{na.rm =}\NormalTok{ T),}
            \AttributeTok{sd\_bps =} \FunctionTok{sd}\NormalTok{(BPSysAve, }\AttributeTok{na.rm =}\NormalTok{ T))}
\end{Highlighting}
\end{Shaded}

\begin{verbatim}
## `summarise()` has grouped output by 'AgeDecade'. You can override using the
## `.groups` argument.
\end{verbatim}

\begin{verbatim}
## # A tibble: 18 x 4
## # Groups:   AgeDecade [9]
##    AgeDecade Gender mean_bps sd_bps
##    <fct>     <fct>     <dbl>  <dbl>
##  1 " 0-9"    female    100.    9.07
##  2 " 0-9"    male       97.4   8.32
##  3 " 10-19"  female    104.    9.46
##  4 " 10-19"  male      110.   11.2 
##  5 " 20-29"  female    108.   10.1 
##  6 " 20-29"  male      118.   11.3 
##  7 " 30-39"  female    111.   12.3 
##  8 " 30-39"  male      119.   12.3 
##  9 " 40-49"  female    115.   14.5 
## 10 " 40-49"  male      121.   14.0 
## 11 " 50-59"  female    122.   16.2 
## 12 " 50-59"  male      126.   17.8 
## 13 " 60-69"  female    127.   17.1 
## 14 " 60-69"  male      127.   17.5 
## 15 " 70+"    female    134.   19.8 
## 16 " 70+"    male      130.   18.7 
## 17  <NA>     female    142.   22.9 
## 18  <NA>     male      136.   23.5
\end{verbatim}

\end{document}
